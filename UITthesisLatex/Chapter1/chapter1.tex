% \pagebreak[4]
% \hspace*{1cm}
% \pagebreak[4]
% \hspace*{1cm}
% \pagebreak[4]

\chapter{Giới Thiệu}
\ifpdf
    \graphicspath{{Chapter1/Chapter1Figs/PNG/}{Chapter1/Chapter1Figs/PDF/}{Chapter1/Chapter1Figs/}}
\else
    \graphicspath{{Chapter1/Chapter1Figs/EPS/}{Chapter1/Chapter1Figs/}}
\fi

\section{Tên đề tài}

\textbf{	NGHIÊN CỨU HỆ THỐNG MOODLE KẾT HỢP CÔNG NGHỆ HTML5 ỨNG DỤNG TRONG VIỆC LUYỆN NGHE MÔN TIẾNG ANH TRÊN MÁY TÍNH BẢNG.}

\section{Đặt vấn đề}

So với cách học tiếng Anh truyền thống như học gia sư, các trung tâm uy tín, và thường gặp nhất là tự học qua sách báo, giáo trình bán sẵn thì việc học tiếng Anh trên thiết bị di động nói chung và chiếc máy tính bảng nói riêng có nhiều ưu điểm như:

\quad - Tranh thủ mọi lúc mọi nơi: Với một chiếc điện thoại hoặc máy tính bảng có cài đặt chương trình luyện nghe, các bạn học viên sẽ hoàn toàn chủ động về thời gian, điều mà các trung tâm không thể nào đem lại được (thậm chí ngay cả việc tự học cũng vậy). Khi đang ngồi trên xe bus hay thậm chí trong lúc đang đi du lịch, chúng ta đều có thể tranh thủ học, nghe và trả lời các câu hỏi tiếng Anh trên thiết bị di động.

\quad - Chi phí phù hợp: Thay vì bỏ ra nhiều tiền để theo những khóa học tại các trung tâm Tiếng Anh hay mua các tài liệu, băng đĩa thì việc sử dụng ứng dụng di động sẽ ít tốn kém hơn nhiều. Đó là điều mà hầu hết các bạn sinh viên đều quan tâm.

\quad - Phương tiện đơn giản: Chiếc điện thoại thông minh hay chiếc máy tính bảng gọn nhẹ đã thực sự rất phổ biến đối với mọi người. Việc cài đặt một ứng dụng hữu ích là bất kỳ ai cũng có thể làm được.

Với những ưu điểm của cách học tiếng Anh qua ứng dụng mà chúng em đã xây dựng, chúng em hy vọng có thể giúp các bạn trẻ, nhất là các bạn học sinh – sinh viên có thể tăng khả năng giao tiếp ngoại ngữ với người nước ngoài, tăng khả năng được tuyển dụng vào những môi trường làm việc quốc tế, đem lại cơ hội phát triển cho bản thân và nguồn thu nhập lớn cho đất nước.

\section{Mục tiêu đề tài}

Hai mục tiêu chính mà nhóm tác giả muốn thực hiện đó là:

\quad - Xây dựng chức năng luyện nghe offline cho ứng dụng.

\quad - Phát triển ứng dụng có khả năng kết nối vào Moodle và thực hiện luyện nghe online.

Dựa trên những công việc của các đối tượng, ứng dụng sẽ có chức năng:

\quad - Chọn hình thức học là luyện nghe hay luyện thi, sau đó nghe bài và trả lời các câu hỏi trắc nghiệm hoặc điền từ vào chỗ trống.

\quad - Đăng nhập vào hệ thống, tải về danh sách các khóa học.

\quad - Đối với từng khóa học, sẽ có các bài luyện tập ở mức độ khác nhau phù hợp cho từng đối tượng khác nhau.

\quad - Lưu các câu trả lời, tính điểm và nhận xét kỹ năng của học viên theo số câu trả lời đúng của từng phần của bài nghe.

\section{Nội dung và giới hạn đề tài}
\subsection{Nội dung đề tài}

	Hiện nay trên thị trường có nhiều loại phần mềm máy tính hỗ trợ việc học ngoại ngữ nói chung và học Anh văn nói riêng, trong đó phần luyện kỹ năng nghe là một thành phần rất quan trọng khi học bất kì một ngoại ngữ nào đó. Tuy nhiên đa phần các phần mềm này còn có nhiều hạn chế sau:
	
	\quad 1. Các ứng dụng này thường chỉ chạy trên một nền tảng nhất định.
	
	\quad 2. Do hạn chế thứ nhất sẽ dẫn đến hạn chế thứ 2 là: khi cần bảo trì nâng cấp thì cần phải thực hiện một cách riêng biệt trên từng nền tảng khác nhau, gây ra tốn kém nhiều về kinh tế và thời gian.
	
	\quad 3. Việc sử dụng phần mềm học ngoại ngữ trên máy tính để bàn hay máy tính xách tay đã không còn phù hợp với xu thế ngày nay, không linh động bằng việc sử dụng các thiết bị di động. Với thiết bị di động, ta có thể học mọi lúc mọi nơi.
	
	\quad 4. Đa số các phần mềm ở dạng offline không có sự tương tác giữa giáo viên và người học.\\
	
	Đề tài này sử dụng công nghệ HTML5 có thể giúp việc phát triển ứng dụng học nghe tiếng Anh chạy trên nhiều nền tảng hệ điều hành như Android, IOS, Windows Phone và việc soạn thảo bài giảng cũng như sự tương tác với giáo viên sẽ do hệ thống Moodle đảm nhiệm.
	
	Đề tài của chúng em sẽ thực hiện các công việc sau:
	
	\quad 1. Nghiên cứu kỹ thuật lập trình trên thiết bị di động dùng công nghệ HTML5.
	
	\quad 2. Nghiên cứu hệ thống Moodle và truyền thông Server-Client.
	

\subsection{Giới hạn đề tài}

	Hệ thống học tập Moodle là một hệ thống quản lý học tập rất phức tạp. Trong phạm vi nghiêm cứu của đề tài này, chúng em chỉ tập trung vào phần soạn câu hỏi trắc nghiệm ở Server và các cấu hình cần thiết cùng phần ứng dụng luyện nghe và trả lời câu hỏi trắc nghiệm ở phía Client.
\newpage
\section{Cấu trúc khóa luận}

	Cấu trúc luận văn này sẽ dàn trải trên 5 chương sau:
	
	\textbf{Chương 1}: Giới thiệu.
	
	\textbf{Chương 2}: Kiến thức nền tảng và khảo sát.
	
	\textbf{Chương 3}: Hệ thống luyện nghe tiếng Anh.
	
	\textbf{Chương 4}: Hiện thực hệ thống luyện nghe tiếng Anh.
	
	\textbf{Chương 5}: Kết luận và hướng phát triển.
