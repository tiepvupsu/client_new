\def\baselinestretch{1}
\chapter{Kết Luận và Hướng Phát Triển}
\ifpdf
    \graphicspath{{Conclusions/ConclusionsFigs/PNG/}{Conclusions/ConclusionsFigs/PDF/}{Conclusions/ConclusionsFigs/}}
\else
    \graphicspath{{Conclusions/ConclusionsFigs/EPS/}{Conclusions/ConclusionsFigs/}}
\fi

\def\baselinestretch{1.66}
 

\section{Kết luận}

Đến đây, chúng em xin đi đến kết luận. Về cơ bản hệ thống của chúng em hầu như hoàn thiện ở phía Client và giúp người sử dụng có khả năng chọn và nghe bài học, trả lời câu hỏi và xem điểm. Còn về phía Server, chúng em chưa hiện thực được ý tưởng một cách đầy đủ, chỉ mới xây dựng được chức năng chọn và tải đề soạn theo cấu trúc XML dựa vào sự hỗ trợ của Web service trên Moodle chứ chưa thật sự tạo ra riêng một hệ thống Web services phục vụ nhu cầu.

Tuy ứng dụng chưa thật sự hoàn chỉnh về mặt chương trình, nhưng nhóm khóa luận tin rằng nếu ứng dụng luyện nghe tiếng Anh này khi hoàn thiện thì khả năng được áp dụng vào thực tiễn rất cao. Chúng em tin rằng ứng dụng sẽ giúp người học nâng cao được kỹ năng nghe hiểu tiếng Anh tốt hơn và thuận tiện hơn.
 
\section{Hạn chế, khó khăn}

Do HTML5 chưa phải là một chuẩn chung của nên nhiều thiết bị còn chưa tương thích và chưa hỗ trợ đầy đủ, nên sẽ có thiết bị sẽ không thể chạy được ứng dụng Client. Qua khảo sát thực tế, chương trình chúng em chạy ổn định trên các thiết bị iPad của Apple và tablet Android.

Trong quá trình phát triển ứng dụng dùng nền tảng IBM Worklight v6, nhóm khóa luận còn gặp trục trặc về nhiều lỗi phát sinh.

\section{Phương hướng phát triển}

Chúng em đề ra phương hướng để phát triển hệ thống học tập như sau:

\quad - Thiết kế lại giao diện để thân thiện hơn với người dùng.

\quad - Thêm chức năng xem từ mới của bài nghe và ghi chú cho người học.

\quad - Cài đặt Web services có khả năng chuyển đổi các câu hỏi trong Quiz Activities trên Moolde thành file XML/JSON, giúp cho việc soạn đề của giáo viên trở nên nhẹ nhàng hơn thay vì phải soạn đề trực tiếp thành file XML.

Trong tương lai xa hơn nữa, hệ thống của chúng em hy vọng sẽ phát triển thành một thư viện e-learning trên máy tính bảng, để có thể tùy chỉnh giảng dạy các môn học khác nhau chứ không chỉ là luyện nghe tiếng Anh với nguồn dữ liệu đồ sộ và xây dựng được một cộng đồng đông đảo, giúp cho việc hoàn thiện chương trình và góp phần nâng cao hơn nữa nền giáo dục nước nhà.

 
%%% ----------------------------------------------------------------------

% ------------------------------------------------------------------------

%%% Local Variables: 
%%% mode: latex
%%% TeX-master: "../thesis"
%%% End: 
